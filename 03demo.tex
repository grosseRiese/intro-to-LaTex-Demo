\documentclass{article}

\usepackage[swedish]{babel}  % Svenska rubriker i innehållsförteckning etc

% Kommentarer inleds med procenttecken och visas inte
% Här kommer en titel och författare att visa i början av dokumnetet
% och i sidhuvudena (om man vill)
\title{LaTex demo}
\author{Some one}
\date{2021-09-29}

\begin{document}

%Skapa titelinfo
\maketitle


%Lägga till rubriker för sektioner etc.
\section{Första sektionen}

Skriv lite text och kompilera.

En tom rad ger nytt stycke. Men nu vill vi väl skriva lite matematik också,
det är ju det som LaTex är verkligt bra på. För att skriva en formel inne i text
skriver man den mellan \textbackslash( och \textbackslash): \( x y z=3 \). Observera
att fonten blir annorlunda när man skriver en formel och dessutom blir det 
inga mellanrum även om man gör mellanslag. Om man vill ha en formel på en egen
rad istället så använder man  \textbackslash [ och \textbackslash ] istället:
\[
x y z = 3.
\]

\subsection{En liten undersektion}
Här finns en del intressant text.


%Avsluta med en innehållsförteckning
\tableofcontents

\end{document}
