\documentclass{article}

\usepackage[swedish]{babel}  % Svenska rubriker i innehållsförteckning etc

\usepackage{amsfonts}  % Innehåller t.ex. fonterna \mathbb
\usepackage{amsmath}  % Innehåller många användara kommandon för matte
\usepackage{amsthm} % Ger omgivningar för att skapa satser och bevis

% Ofta vill man definiera egna kommandon för sådana saker som man skriver ofta
\newcommand{\tb}{\textbackslash}  % Gör att den late kan skriva bara \tb
\newcommand{\R}{\mathbb{R}}       % Ger reella talen
\newcommand{\Z}{\mathbb{Z}}       % Ger heltalssymbolen

% Skapa olika typer av satser som man vill använda i sin fil (kräver amsthm)
% Det första är namnet på omgivningen som använder när man skapar den
% och det andra är det som skrivs ut.
\newtheorem{sats}{Sats}
\newtheorem{prop}{Proposition}
\newtheorem{lemma}{Lemma}

% Kommentarer inleds med procenttecken och visas inte
% Här kommer en titel och författare att visa i början av dokumnetet
% och i sidhuvudena (om man vill)
\title{LaTex demo}
\author{author name}
\date{2021-09-29}

\begin{document}

%Skapa titelinfo
\maketitle


%Lägga till rubriker för sektioner etc.
\section{Första sektionen}

Skriv lite text och kompilera.

En tom rad ger nytt stycke. Men nu vill vi väl skriva lite matematik också,
det är ju det som LaTex är verkligt bra på. För att skriva en formel inne i text
skriver man den mellan \textbackslash( och \textbackslash): \( x y z=3 \). Observera
att fonten blir annorlunda när man skriver en formel och dessutom blir det 
inga mellanrum även om man gör mellanslag. Om man vill ha en formel på en egen
rad istället så använder man  \textbackslash [ och \textbackslash ] istället:
\[
x y z = 3.
\]

Nu kan man snabbt skriva både reella talen och heltalen \( \R\Z\ \). 

Här ska vi nu se exempel på hur man skapar satser med hjälp av de kommandon
\tb newtheorem vi la till i inledningen av filen.

\begin{sats}
(Aritmetikens fundamentalsats) Varje heltal \(n\geq 2\) kan skrivas som
en unik produkt av primtal.
\end{sats}

\begin{proof}
Vi delar upp beviset i två delar:  etc
\end{proof}

\begin{prop}
En liten proposition är också en sats.
\end{prop}

Observera att texten i satserna är kursiv. Detta är standard när man typsätter
matematisk text. Om man inte gillar det så kan man förstås ändra det (överkurs
för den hågade).

\begin{lemma}
Hjälpsatser kan man inte få för många av.
\end{lemma}

\subsection{En liten undersektion}
Här finns en del intressant text. Nu kommer det en liten kavalkad av olika
matematiska tecken som man använder ofta. Jämför helt enkelt koden nedan med
resultatet du får när du kompilerar. För att få exponenter och index använder
man sig av \(x_{a}^{2} + x_{b}^{2} = z^{10}\). Om det bara är en siffra eller
bokstav kan du strunta i krullparenteserna så \(x_a^2 + x_b^2 = z^{10}\)
ger samma sak. För att få rätt font för olika standardfunktioner som sinus etc ska man använda de
speciella kommandon som finns så t.ex. 
\[
\forall x\in\mathbb{R}\; \sin^2(x)+\cos^2(x) = 1.
\]
Kommandot \textbackslash; gav ett mellanrum. Det finns olika kommandon för 
olika långa mellanrum.

Här kommer en fin formel som innehåller lite fler exempel på matematiska
symboler du skulle kunna vara sugen på att använda
\[
\sum_{i=1}^n \frac{\alpha^i-\beta^i}{\gamma - \delta} = \prod_{j=1}^{2n}
\log(4j + j^2) f(j).
\]

För att få nummer på sina ekvationer så använder man omgivningen ``equation''
och man kan ge den ett namn för att referera till den med kommandot 
\textbackslash label:
\begin{equation}
  \label{eq:nummer1}
\sum_{i=1}^n \frac{\alpha^i-\beta^i}{\gamma - \delta} = \prod_{j=1}^{2n}
\log(4j + j^2) f(j).  
\end{equation}

Man kan sedan referera till denna med kommandot \textbackslash eqref såhär:
ett fint exempel på enkel formel är \eqref{eq:nummer1}.

Ibland blir formlerna så långa att de inte får plats på en rad och då kan
man dela upp dem med kommandot \textbackslash split. Radbrytning markeras 
med\footnote{Exempel på en fotnot: 
I löpande text ger \textbackslash\ följt av 
mellanslag ett normalt mellanrum och det måste användas efter kommandon.}
\textbackslash\textbackslash\ och hur de ska anpassas horisontellt anges med
\&. Såhär kan det se ut:
\begin{equation}
  \begin{split}
  f(p+1)-g(p+1) &= \left(f(p)+\frac1{(p+1)^2}\right)-g(p+1) \\
  &< g(p)+\frac1{(p+1)^2}-g(p+1) \\
  &= \left(2-\frac1p\right)+\frac1{(p+1)^2} -\left(2-\frac1{p+1}\right) \\
  &=-\frac1p+\frac1{(p+1)^2}+\frac1{p+1} \\
  &= \frac{-(p+1)^2+p+p(p+1)}{p(p+1)^2} \\
  &= \frac{-p^2-2p-1+p+p^2+p}{p(p+1)^2} \\
  &= \frac{-1}{p(p+1)^2}<0.    
  \end{split}
\end{equation}

Om man skulle vilja ha nummer på varje rad kan man använda \textbackslash align
istället
\begin{align}
  f(p+1)-g(p+1) &= \left(f(p)+\frac1{(p+1)^2}\right)-g(p+1) \\
  &< g(p)+\frac1{(p+1)^2}-g(p+1) \\
  &= \left(2-\frac1p\right)+\frac1{(p+1)^2} -\left(2-\frac1{p+1}\right) \\
  &=-\frac1p+\frac1{(p+1)^2}+\frac1{p+1} \\
  &= \frac{-(p+1)^2+p+p(p+1)}{p(p+1)^2} \\
  &= \frac{-p^2-2p-1+p+p^2+p}{p(p+1)^2} \\
  &= \frac{-1}{p(p+1)^2}<0.    
\end{align}
Observera att split ska vara inuti en matematikomgivning medan align är en 
egen matematikomgivning.


%Avsluta med en innehållsförteckning
\tableofcontents

\end{document}