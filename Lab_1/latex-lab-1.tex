\documentclass{article}

\usepackage[swedish]{babel}  % Svenska rubriker i innehållsförteckning etc
\usepackage[a4paper]{geometry}
\usepackage[usenames,dvipsnames]{color} %using the color package, not xcolor
\usepackage{amssymb}
\usepackage{amsfonts}  % Innehåller t.ex. fonterna \mathbb
\usepackage{amsmath}  % Innehåller många användara kommandon för matte


\title{LaTex Lab-1}
\author{OMRAN ABUDAHER}
%\date{2021-09-29}
\date{\today}

\begin{document}

%Skapa titelinfo
\maketitle

%Lägga till rubriker för sektioner etc.
\section{Induktionsbevis}
\subsection{Strategi för induktionsbevis} \

\begin{flushleft}

 Om man vill visa att ett påstående är sant för alla heltal \(n\geq a\), så kan man göra på följande sätt:

 \begin{itemize}
  \item \textcolor{Sepia}{\textbf{Induktionsbas: }} Visa att påståendet är sant för  \(n =a\).
  
  \item \textcolor{Sepia}{\textbf{Induktionsantagande: }} Anta att påståendet är sant för ett visst värde på n, t.ex. \(n = p\), dvs. anta att \(VL_{p} = HL_{p}\)
  
  \item \textcolor{Sepia}{\textbf{Induktionssteg: }} Visa med hjälp av induktionsantagandet att påståendet där är sant även för \(n = p+1\), dvs.att \(VL_{p+1} = HL_{p+1}\)

  \item \textcolor{Sepia}{\textbf{Slutsats: }} Eftersom påståendet är sant för \(n = a\) och för två på varandra följande tal, så är påståendet sant för alla tal \(n\geq a\).
  
\end{itemize}

\end{flushleft}


\subsection{Uppgiften: } 

\begin{flushleft}
 Visa att : \[
  \sum_{k=1}^n \frac{1}{k(k+1)} = \frac{n}{n+1}.
  \]
\end{flushleft} 

 \section*{\underline{\textcolor{Brown}{\textit{ Min lösning för denna upgift så här:}} }}

  \begin{itemize}
    \item För $n=1$, får vi (induktionsbas): \
    
      $VL =$ \[
        \sum_{k=1}^n \frac{1}{k(k+1)} = \frac{1}{1(1+1)} = \frac{1}{2}
        \]
        $ HL = $
        \[\frac {n}{n+1} = \frac{1}{1+1}= \frac{1}{2}
        \]
        $\therefore VL= HL ,\> \> \> \>  \forall n\in\mathbb{Z_{+}} $\
    
    \item Nu antar vi att påståendet är sant för ett visst värde av $n$, vi kallar det värdet för $p$. Alltså antar vi att för $n=p$ gäller $\textcolor{ForestGreen}{\frac{1}{p(p+1)} } = \frac{p}{p+1}$. (induktionsantagande)
    \begin{equation}
      \begin{split}
      \end{split}
    \end{equation}

    \item Vi ska nu undersöka om detta medför att påståendet är sant även för nästa värde på $n$,dvs. för $ n=(p+1) $ : \\
    Vi sätter in $ (n = p+1)$ i HL.
    
    \[
      \textcolor{BrickRed}{ VL_{(p+1)} } = \> \>
      \sum_{k=1}^{n+1}\> \textcolor{ForestGreen}{\frac{1}{p(p+1)}}+\frac{1}{(p+1)(p+2)} = (\frac{p+1}{p+2}) = \textcolor{BrickRed}{ HL_{(p+1)} }
    \]
  
    Från $(1)$ (induktionsantagande):
    
    $
    \textcolor{BrickRed}{ VL_{(p+1)} }
    $
    \[
      = \textcolor{ForestGreen}{ \frac{p}{p+1}}+\frac{1}{(p+1)(p+2)}
    \]
    \[
      = \frac{p(p+2)+1}{(p+1)(p+2)}
    \]
    \[
      = \frac{p^2+2p+1}{(p+1)(p+2)}
    \]
    \[
      = \frac{\textcolor{Green}{(p+1)}(p+1)}{\textcolor{Green}{(p+1)}(p+2)}
    \]
    \[
      = \frac{(p+1)}{(p+2)} \> \> \textcolor{BrickRed}{ = HL_{(p+1)} }
    \]
    Alltså är $ \textcolor{BrickRed}{ VL_{(p+1)} = HL_{(p+1)} }$ , dvs.$VL= HL$ även för $p+1$.
    
    \item Jag har nu visat att påståendet är sant för $n=1$, så är det också sant för nästa tal $p+1$. Detta gör att vi kan dra slutsatsen att påståendet också är sant för alla positiva heltal\ 
    $$\therefore \forall n\in\mathbb{Z_{+}} $$
  \end{itemize}

\end{document}