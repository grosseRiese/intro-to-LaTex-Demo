\documentclass{article}

\usepackage[swedish]{babel}  % Svenska rubriker i innehållsförteckning etc

\usepackage{amsfonts}  % Innehåller t.ex. fonterna \mathbb
\usepackage{amsmath}  % Innehåller många användara kommandon för matte

% Kommentarer inleds med procenttecken och visas inte
% Här kommer en titel och författare att visa i början av dokumnetet
% och i sidhuvudena (om man vill)
\title{LaTex demo}
\author{Someone else}
\date{2021-09-29}

\begin{document}

%Skapa titelinfo
\maketitle


%Lägga till rubriker för sektioner etc.
\section{Första sektionen}

Skriv lite text och kompilera.

En tom rad ger nytt stycke. Men nu vill vi väl skriva lite matematik också,
det är ju det som LaTex är verkligt bra på. För att skriva en formel inne i text
skriver man den mellan \textbackslash( och \textbackslash): \( x y z=3 \). Observera
att fonten blir annorlunda när man skriver en formel och dessutom blir det 
inga mellanrum även om man gör mellanslag. Om man vill ha en formel på en egen
rad istället så använder man  \textbackslash [ och \textbackslash ] istället:
\[
x y z = 3.
\]

\subsection{En liten undersektion}
Här finns en del intressant text. Nu kommer det en liten kavalkad av olika
matematiska tecken som man använder ofta. Jämför helt enkelt koden nedan med
resultatet du får när du kompilerar. För att få exponenter och index använder
man sig av \(x_{a}^{2} + x_{b}^{2} = z^{10}\). Om det bara är en siffra eller
bokstav kan du strunta i krullparenteserna så \(x_a^2 + x_b^2 = z^{10}\)
ger samma sak. För att få rätt font för olika standardfunktioner som sinus etc ska man använda de
speciella kommandon som finns så t.ex. 
\[
\forall x\in\mathbb{R}\; \sin^2(x)+\cos^2(x) = 1.
\]
Kommandot \textbackslash; gav ett mellanrum. Det finns olika kommandon för 
olika långa mellanrum.

Här kommer en fin formel som innehåller lite fler exempel på matematiska
symboler du skulle kunna vara sugen på att använda
\[
\sum_{i=1}^n \frac{\alpha^i-\beta^i}{\gamma - \delta} = \prod_{j=1}^{2n}
\log(4j + j^2) f(j).
\]

%Avsluta med en innehållsförteckning
\tableofcontents

\end{document}
